\chapter{Discussion}

\section{Conclusions \& Future Work}

In this thesis, we have shown the procedure followed in order to build a classifier using a logistic regression model. The main objective of the work was to mine knowledge from the on-line platform \emph{Decidim Barcelona} and the process that took place to decide which proposals were going to be accepted and which not. To do so first an analysis was performed to the whole dataset to get a better understanding at the building of the model. This prepossessing already showed some indicators of which variables could influence the decision. Then when we moved on to the building and understanding of the model itself, we saw the same patterns that were already visible in the previous analysis, but when we compared it against a more complex model, our classifier showed to work not as good, but the most remarkable fact was that each model had different features as main drivers of the classification. On our model, categorical variables had a larger weigh in importance while the Random Forest performed had as primary features the numerical ones tied to the proposal. We learned that the variables used lacked some key information to the decision making process, not just because a proposal is really commented and praised by the community this will imply an action will come out of it and will be accepted. Here political competences and limitations take over, and is where our models are blind. This could be tried to overcome by introducing richer text features, mining the information from title, text and comments from proposals. The results we saw where all interaction features had low importance on the logistic regression model versus more structural ones like the district or the origin of the proposals could be discouraging since it disconnects users' interaction from the decision of acceptance / rejection. Nevertheless, this process was the first of its kind in the city of Barcelona, and its a promising step towards e-democracy. The results we observe are quite biased since most of the proposals are coming directly from the city hall, dumped into the platform and accepted a posteriori. This does not mean that in the near future, if the platform works as its intended to, with a more pure origin of proposals, more insights could be draw out from it repeating a similar analysis and model.

\newpage


