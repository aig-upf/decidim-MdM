%\addcontentsline{toc}{chapter}{Abstract}

\begin{abstract}
\pagenumbering{gobble}

Decidim Barcelona is the citizen engagement platform that Barcelona's city hall put in motion on February 2016. One of  its first processes has been the development of the "Pl\`a d'Actuaci\'o Municipal" (PAM) which has led to the creation of $10,860$ proposals that eventually resulted in $1,467$ actions included in the plan.

In this project we build a statistical model with the aim to predict whether an individual proposal would be accepted or not. The objectives of this work are two-fold: first, we want to gain understanding in the processes that took place during the development of the PAM and second, we want to test to what extent this type of statistical modeling can capture the decision making process in order to better aid future deliberative processes in the platform.

We first analyze the data generated by citizens in the city of Barcelona that participated actively on the platform. After a preliminary statistical analysis of the features that characterize each proposal, we proceed to build a model that is able to predict if a proposal would be accepted or not from that data. We consider the logistic regression model because its computational simplicity as well as its potential interpretability. We are be able to extract conclusions from the parameters and unveil the decision process which resulted the acceptance/rejection of each proposal in the platform. We show that such a model is able to characterize some particularities of the process, but also how this classifier compares to other methods like random forests, and what do the differences we find between them mean.

\bigskip
Keywords: Logistic Regression Classifier; Collective Action; e-Democracy


\newpage
\end{abstract}
